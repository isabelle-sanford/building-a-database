% source/licensing info at the end

%----------------------------------------------------------------------------------------
\documentclass[12pt, oneside]{amsart} % Uses amsart class in A4 format

\setlength{\parskip}{0pt}
\setlength{\parindent}{0pt}
\setlength{\voffset}{-15pt}

%----------------------------------------------------------------------------------------
%	PACKAGES AND OTHER DOCUMENT CONFIGURATIONS
%----------------------------------------------------------------------------------------
% formatting
\usepackage[letterpaper, margin=1in]{geometry} % Sets margin to 1in
\usepackage[onehalfspacing]{setspace} % Sets Spacing to 1.5
\usepackage[utf8]{inputenc} % Use UTF-8 encoding
\usepackage[all]{nowidow} % Removes widows
\usepackage{parskip} % what does this do???
\usepackage{datetime2} % Uses YEAR-MONTH-DAY format for dates

% headers & footers-----------
\usepackage{fancyhdr} % Headers and footers
\pagestyle{fancy} % All pages have headers and footers
\fancyfoot[R]{Isabelle Sanford} 
\fancyfoot[L]{\thepage}
\fancyfoot[C]{} % blank out the default page number

% images & urls
\usepackage[final, colorlinks = true, 
            linkcolor = blue, 
            citecolor = black,
            urlcolor = blue]{hyperref} % For hyperlinks in the PDF
\usepackage{graphicx, multicol} % Enhanced support for graphics

% math stuff setup
\usepackage[english]{babel} % Language hyphenation and typographical rules

\usepackage{amsthm, amsmath, amssymb, stix} % Mathematical typesetting
\usepackage[]{algorithm2e}


% Definition, Theorem, Lemma, Exercise, Reflection, Proposition, Corollary
\newenvironment{problem}[2][Problem]{\begin{trivlist}
\item[\hskip \labelsep {\bfseries #1}\hskip \labelsep {\bfseries #2.}]}{\end{trivlist}}

\newtheorem*{theorem}{Theorem}
\newtheorem*{lemma}{Lemma}

%----------------------------------------------------------------------------------------

\begin{document}

%----------------------------------------------------------------------------------------
%	TITLE SECTION
%----------------------------------------------------------------------------------------

\title{Senior Project Proposal} % Article title

\begin{minipage}{0.295\textwidth} % Left side of title section
\raggedright
CMSC 399\\ % Your lecture or course
\footnotesize % Authors text size
%\hfill\\ % Uncomment if right minipage has more lines
Senior Seminar % Your name
\medskip\hrule
\end{minipage}
\begin{minipage}{0.4\textwidth} % Center of title section
\centering 
\large % Title text size
Database Project\\ % title
\normalsize 
Proposal \\ % subtitle
\end{minipage}
\begin{minipage}{0.295\textwidth} % Right side of title section
\raggedleft
Isabelle Sanford \\% Date
\footnotesize 
%\hfill\\ % Uncomment if left minipage has more lines
isanford@brynmawr.edu
\medskip\hrule
\end{minipage}\\
%--------------------------------------------------------------------------------------

\section{Summary}

[1 sentence summary]

This project is to build a simple version of a database engine, which can be modified and evaluated easily to understand and test the use of various common database features. 


\section{Problem Statement}
Databases are a vital part of everyday life, and something that many programmers will have to interact with regularly throughout their careers. But modern databases are practically inscrutable, because of the level of complexity required for commercial-level databases (ehhh). This isn't usually a problem for the end user, if the user interface is designed well, but there are people who need to actually understand what's going on and how these things work. That's people like database admins, data scientists, and the database engineers who make it all in the first place. 

Education(?) about database internals is analagous to education about operating systems - the theory isn't too hard to teach, but practical examples are more difficult. Looking at the internals of a modern database is too much, like trying to teach highschool level chemistry using a graduate-level book. But teaching individual parts of a database is hard because of how intertwined it is. So what's needed is a small, simple 'model' database engine with clear (and alterable!) internals. The alterable point is important: a professor could have pieces of example code of a model database in lectures, but that's still not practical learning like actually using and changing the code yourself is. 

\section{Proposed Solution}

This project is a simple (though functional) database, designed to have most of the basic features of a modern database, but implemented in a clearer and more naive way. It's written in Go, a language which would be reasonable for this kind of database to be written in. 

Features include (but are not limited to; more may be added on as the project progresses):
\begin{itemize}
    \item File Manager can read and write pages directly to the disk
    \item In-use pages are cached in a buffer pool 
    \item Changes to the database are logged as they occur, which happens in set distinct transactions
    \item Records are stored in tables in a such a way that the engine knows where and how they're organized (via a catalog of the tables and their metadata)
    \item Stat Manager keeps track of statistics about tables (e.g. the number of records) in order to plan queries to be more effective 
    \item The engine can take (simple) SQL input from the user and perform the appropriate relational algebra to return a (correct) result
\end{itemize}



\section{Evaluation Plan}

The success of this project, beyond the basic functionality of the database, depends on whether the engine is easy to alter and the alterations are easy to test in order to see results. So the evaluation plan is simply to try a bunch of tests changing the database engine and see how easy it is to do correctly and get results. (TBD: exactly what results this means, but it'll probably include time, number of operations, number of buffer pages used or read/writes, or something along those lines.) 

Possible tests are being compiled as the database is built, but potential options include: 
\begin{itemize}
    \item The order in which the buffer manager picks unpinned buffer pages to overwrite (affecting the number of reads/writes)
    \item The stats (both which stats and how precise(?)) which the stats manager uses to plan queries
    \item How records are stored in files/tables (e.g. kept inside one block or able to span multiple, fixed or variable length, kept in one file or several)
\end{itemize}


\section{Potential Challenges}

Making a database engine is hard! 

\section{Bio}

\section{Elevator Pitch}


\end{document}





%%%%%%%%%%%%%%%%%%%%%%%%%%%%%%%%%%%%%%%%%
% Homework Assignment Article LaTeX Template
% Version 1.3.5r (2018-02-16)
% This template has been downloaded from: /cl.uni-heidelberg.de/~zimmermann/
%
% Original author: Victor Zimmermann (zimmermann@cl.uni-heidelberg.de)
% Modified and used by Isabelle Sanford
%
% License: CC BY-SA 4.0 (https://creativecommons.org/licenses/by-sa/4.0/)
%%%%%%%%%%%%%%%%%%%%%%%%%%%%%%%%%%%%%%%%%